\subsection*{2--1}
With banlance of mass law in spherical coordiantes, we have:
$\frac{\partial\rho}{\partial t}+\frac1{r^2} \frac{\partial(\rho r^2 v_r)}{\partial r} =0$ (Eliminate the components of $v_{\theta}$ and $v_{\phi}$) \\
Thus, we would have 
\begin{align*}
    \rho_{,t}+\frac1{r^2}[\rho_{,r}r^2 v_r+2\rho r v_r +\rho r^2 v_{r,r}]&=0 \\
    \rho_{,t}+ \rho_{,r} v_r + \frac{\rho}{r} (2v_r + r v_{r,r}) &= 0 \\
\end{align*}
For the assumption of incompressibility, we assume that $\rho$ is constant, thus we have 
\begin{align*}
    2v_r + rv_{r,r}&=0 \\
    \frac{2\partial r}{r} &= - \frac{\partial v_r}{v_r} \\
    2 \ln{r}&= -\ln{v_r} + C 
\end{align*}
For the boundary condition, we have: when $r=R(t),\ v_r = \dot{R}$.\\
Thus,
\begin{align*}
    2 \ln{R} &= -\ln{\dot{R}} +C\\
    C&= \ln{R^2 \dot{R}}
\end{align*}
Hence,
\begin{align*}
    2 \ln{r}&= -\ln{v_r} +\ln{R^2 \dot{R}} \\
    \ln{v_r} &= \ln{\frac{R^2 \dot{R}}{r^2}} \\
    v_r(r,t) &= \frac{R^2 \dot{R}}{r^2}
\end{align*}

\subsection*{2--2}
The force of the surface traction from water is
\begin{equation*}
    \bm{F_w} = \int\limits_{\partial \mathcal{B}} -\rho_w g x_3 \hat{\bm{n}}dA =-\rho_w g \int \limits_\mathcal{B}\nabla{x_3}dV =-\rho_w g V\bm{e_3} = -4/3\pi p_wgR^3 \bm{e_3}
\end{equation*} \\
The gravitational force is
\begin{align*}
    \bm{F_g}& = \int \limits_{\mathcal{B}}\rho(\bm{x})\bm{g}dV \\
    &=\bm{g} 
    \int^{R}_{-R}\pi (R^2-x^2_2)(\frac{\rho_w}2 +\frac{\rho_w}{2R} x_2)dx_2  \\
    &= \frac23 \pi \rho_w R^3g \bm{e_3}
\end{align*}
Thus, the net force is $\bm{F_{net}} = \bm{F_w} +\bm{F_g} = -\frac23 \pi \rho_w R^3g \bm{e_3}$ \\

Let's say the center of the sphere is $\bm{x_c}=(x_c,y_c,z_c)$, each point is distanced from it with a vecotr $\bm{r}=(x_1,x_2,x_3)$. Thus, the moment from water traction is 
\begin{align*}
     \bm{J_w} &= \int\limits_{\partial \mathcal{B}} -\rho_w g  \bm{r} \times (z_c+x_3) \hat{\bm{n}}dA \\
     &=  -\rho_w g\int\limits_{\partial \mathcal{B}}(z_c+x_3)(\bm{r}\times\hat{\bm{n}})dA  \\
     &= 0
\end{align*}
$\bm{r}\times\hat{\bm{n}}=0$ since $\hat{\bm{n}}=\frac{\bm{r}}{|\bm{r}|}$ and $\bm{r}\times\bm{r}=0$ \\
For the angular momentum from gravity,
\begin{align*}
    \bm{J_g}&= 
    \int\limits_{\mathcal{B}} \bm{r} \times (\frac{\rho_w}2 +\frac{\rho_w}{2R} x_2)g\hat{\bm{e_3}}dV  \\
    &=-\rho_w g \int\limits_{\mathcal{B}}(\frac12 +\frac{x_2}{2R})( \bm{r} \times \hat{\bm{e_3}})dV\\
    &=-\rho_w g \int\limits_{\mathcal{B}}(\frac12 +\frac{x_2}{2R})
    \begin{pmatrix}
        -x_2 \\x_1 \\0
    \end{pmatrix}dV
\end{align*} \\
For the x component, the integral simplifies to $\rho_w g \int\limits_{\mathcal{B}}\frac{x_2^2}{2R}dV$ since the other term is odd function across V. Then the integral goes to
\begin{align*}
    \rho_w g \int\limits_{\mathcal{B}}\frac{x_2^2}{2R}dV
    &=\rho_w g \int^{R}_{-R}\frac{x_2^2}{2R}\pi(R^2-x_2^2)dx_2\\
    &=\frac2{15}\pi\rho_w g R^4 \\
    \end{align*}
As for the y component, since it's also odd function across V, thus it's 0. \\
Therefore, the moment of fravity force is $(\frac2{15}\pi\rho_w g R^4,0,0)$. The net moment is $(\frac2{15}\pi\rho_w g R^4,0,0)$.

\medskip
(b)
1) The buoyancy force should balance the gravity, so the only volume that the ${\mathcal{B}}$ can be submerged is $ \frac{\bm{F_g}}{\rho_{\omega}g\bm{e_3}}= \frac23 \pi R^3$, only $\frac12$ of the sphere can be submerged.\\
2) The angular momentum should be zero. So the gradient of the density should on the $\hat{\bm{e_3}}$ direction, so that in the integral $\int\limits_{\mathcal{B}}(\frac12 +\frac{x_3}{2R})
    \begin{pmatrix}
        -x_2 \\x_1 \\0
    \end{pmatrix}dV$ equals 0.


\subsection*{2--3}

As in Chapter 3 in J.D. Ferry's book\citep{ferry1980viscoelastic} and Chapter 8 in  Tschoegl's\citep{tschoegl2012phenomenological} that $G_r(t)$ and $J_c(t)$ hold the relationship of  $G_r(t)J_c(t)\leq1$, we can not simply assume this stress relaxation spectrum can give us any information about creep spectrum. However, without more information we can estimate the creep spectrum with this plot still.\\
(a) Yes it can be called a material with linear viscoelasticity. when the strain is from 0 to $2 \times 10^{-3}$, they are in linear viscoelasticiy regime. \\
\\
(b) When $\sigma = 100$ Pa, the strain is about 0.43, 0.52, 0.66,0.81,1.0$\times 10^{-3}$  at 2, 5, 10, 20, 40 s, which  gives a Jc of 0.43, 0.52, 0.66,0.81,10 $\times10^{-6} Pa^{-1}$. After the curve fitting, we have a formula $J_c(t) =1.542\times10^{-6}+2.060\times10^{-6}t^{0.384}$. \\
When $\sigma = 250$ Pa, the strain is about 1.2,1.45, 1.80, 2.40, 3.00,$\times 10^{-3}$  at 2, 5, 10, 20, 40 s, which  gives a Jc of 4.80, 5.80, 7.20, 9.60, 12.0 $\times10^{-6} Pa^{-1}$. After the curve fitting, we have a formula $J_c(t) =2.463\times10^{-6}+1.584\times10^{-6}t^{0.489}$. 
\begin{figure}[htbp]
\centering
\includegraphics[width=0.7\linewidth]{HW2.jpg}
\caption{Fig.1 Fitting Curves}
%\label{fig:my_label}
\end{figure}
\medskip
\subsection*{2--4}
(a) 
For a general function, 
\begin{equation*}
    \epsilon(t) = \int^t_0 J_c(t-\tau)\frac{d(A\delta(\tau))}{d\tau}d\tau = \int^t_0 J_c(t-\tau) A \psi(\tau)d\tau
\end{equation*} \\
For a K-V model, we have
\begin{equation*}
    \epsilon(t) = \frac{A\delta(t=0^-)}{E}(1-\exp(-Et/\eta)) + \frac1E \int^t_{0^-}[1-\exp(-E(t-\tau)/\eta)]A\psi(\tau)d\tau
\end{equation*} \\
Do the Laplace transform on the second term $I_2$, we get
\begin{align*}
    \mathcal{L}\{I_2(t)\}&=\frac1E (\frac1s -\frac1{s+E/\eta})\cdot As \\
    &=\frac{A}{\eta(s+E/\eta)} \\
    \mathcal{L}^{-1}\{I_2(s)\}&=\frac A \eta \exp{(-Et/\eta)}
\end{align*} \\
Thus, the final strain is 
\begin{equation*}
    \epsilon(t) = \frac{A\delta(t=0^-)}{E}(1-\exp(-Et/\eta))+\frac A \eta \exp{(-Et/\eta)}
\end{equation*}
\\
(b)
For a general function, 
\begin{equation*}
    \epsilon(t) = \int^t_0 J_c(t-\tau)\frac{d(B\psi(\tau))}{d\tau}d\tau = \int^t_0 J_c(t-\tau) B \frac{\partial \delta(\tau)^2}{\partial^2 \tau}d\tau
\end{equation*} \\
For a K-V model, we have
\begin{equation*}
    \epsilon(t) = \frac{B\psi(t=0^-)}{E}(1-\exp(-Et/\eta)) + \frac1E \int^t_{0^-}[1-\exp(-E(t-\tau)/\eta)]B \frac{\partial \delta(\tau)^2}{\partial^2 \tau}d\tau
\end{equation*} \\
Do the Laplace transform on the second term $I_2$, we get
\begin{align*}
    \mathcal{L}\{I_2(t)\}&=\frac1E (\frac1s -\frac1{s+E/\eta})\cdot Bs^2 \\
    &=\frac{Bs}{\eta(s+E/\eta)} \\
    &=\frac B\eta(1-\frac{E}{\eta(s+E/\eta)}) \\
    \mathcal{L}^{-1}\{I_2(s)\}&=\frac B \eta [\delta(t)-\frac E{\eta}\exp{(-Et/\eta)}]
\end{align*} \\
Thus, the final strain is 
\begin{equation*}
    \epsilon(t) = \frac{B\psi(t=0^-)}{E}(1-\exp(-Et/\eta))+\frac B \eta [\delta(t)-\frac E{\eta}\exp{(-Et/\eta)}]
\end{equation*}

\medskip
\subsection*{2--5}
(a) $\epsilon(t)=d+d\sin(\omega t)$
\begin{align*}
    \sigma(t) &= G(t)\epsilon(0^+)+\int^t_{0^+}G(t-\tau)\dot\epsilon(\tau)d\tau \\
    &=(E+\eta\delta(t))d+\int^t_{0^+}[E+\eta\delta(t-\tau)]\dot\epsilon(\tau)d\tau \\
    &= (E+\eta\delta(t))d+Ed\sin(\omega t) +\eta d \omega\cos (\omega t) \\
    &= Ed(1+\sin(\omega t))+\eta d(\delta(t) + \omega \cos(\omega t)) \\
\end{align*}
\\
(b) Except for the initial response, the stress turns to $Ed\sin(\omega t)+\eta\omega d\cos(\omega t) $
\begin{equation*}
    \sin(\omega t+\delta)=\sin(\omega t)\cos \omega+\cos(\omega t) \sin \delta
\end{equation*} \\
Thus, $\sin\delta =\eta \omega d,\ \cos \delta = Ed ,\ \tan\delta=\frac{\eta\omega}{E}$ \\
\\
(c) 
\begin{align*}
    \epsilon(t) &= \frac{\sigma(t=0^-)}{E}(1-\exp{(-Et/\eta)})+\frac1E\int^t_{0^-}[1-\exp(-E(t-\tau)/\eta)]\dot\sigma(\tau)d\tau \\
    &= \frac{-\sigma_0}{E}(1-\exp{(-Et/\eta)})-\frac1E\int^t_{0^-}[1-\exp(-E(t-\tau)/\eta)]\omega \sigma_0 \cos{(\omega \tau)}d\tau  \\
    &= \frac{-\sigma_0}{E}(1-\exp{(-Et/\eta)})-\frac{\sigma_0}E\sin{(\omega t})+\frac{\omega \sigma_0}{E}\int^t_{0^-}\exp(-E(t-\tau)/\eta) \cos{(\omega \tau)}d\tau \\ 
\end{align*}
Let the third term on the right hand side be $I(t)$ and do tha laplace transform of it

\begin{align*}
    \mathcal{L}\{I(t)\}&=\frac1{(s+E/\eta)}\frac s{(s^2+\omega^2)} \\
    &= \frac{A}{(s+E/\eta)}+\frac {Bs+C}{(s^2+\omega^2)} \\
\end{align*}
By combining and comparing the same terms, we get
\begin{equation*}
    A= -\frac{E}{\eta D},\ B=\frac{E}{\eta D},\ C= \frac{\omega^2}{D} \\ 
\end{equation*}
where $D=\omega^2+\frac{E^2}{\eta^2}$\\
Then do the inverse Laplace transform
\begin{align*}
    \mathcal{L}^{-1}\{I(s)\}&=\mathcal{L}^{-1}\{-\frac{E}{\eta D(s+E/\eta)}+\frac{Es}{\eta D(s^2+\omega^2)}+\frac{\omega^2}{D(s^2+\omega^2)} \} \\
    &=-\frac{E}{\eta D}\exp{(-\frac E\eta t)}+\frac{E}{\eta D}\cos{(\omega t)}+\frac{\omega}{D}\sin{(\omega t)}
\end{align*}
\\
Thus, 
\begin{equation*}
    \epsilon(t) = -\frac{\sigma_0}E+(\frac{\sigma_0}{E}-\frac{E}{\eta D})\exp{(-\frac E\eta t)}+\frac{\omega \sigma_0}{\eta D}\cos{(\omega t)}+(\frac{\omega^2\sigma_0}{DE}-\frac{\sigma_0}E)\sin{(\omega t)}\\
\end{equation*} 
where $D=\omega^2+\frac{E^2}{\eta^2}$\\
\\
(d) Follow the same analysis of Part (b), when we reach steady state $(t \rightarrow\infty)$, we have
\begin{equation*}
    \tan \delta = \frac{\frac{\omega \sigma_0}{\eta D}}{\frac{\omega^2\sigma_0}{DE}-\frac{\sigma_0}E}=-\frac{\omega \eta}{E}
\end{equation*}
which is the same as in Part (b) but inverse sign.

\bigskip

\textbf{Part II Project} 
\subsection*{\textbf{1. Introduction}}
\begin{itemize}
\item While the term "hemorheology" was first coined in 1951 \citep{stoltz2017history}, research into the flow behavior of blood began decades earlier \citep{fahraeus1931viscosity}\citep{errill1969rheology}. The non-Newtonian nature of blood flow has attracted significant interest, driving numerous research efforts in the field.
\item With over 5\% of the U.S. population affected by cardiovascular disease \citep{martin20252025}, hemorheological markers such as blood viscosity \citep{de2005association} and red blood cell (RBC) deformability \citep{mbah2024blood} are closely associated with clinical blood-related symptoms.
\end{itemize}

\subsection*{\textbf{2. State of the Field}}
\begin{itemize}
\item Numerous experimental techniques have been utilized to investigate hemorheology, including methods that assess RBC deformability \citep{schmid1969fluid}\citep{baskurt2003blood}, RBC aggregation \citep{hardeman1994laser}, and microcirculation \citep{popel2005microcirculation}.
\item Moreover, theoretical analysis and computational simulations are widely applied. For instance, several constitutive equations have been proposed to describe the non-Newtonian properties of blood \citep{sequeira2018hemorheology}\citep{zhang2000study}. Various simulation techniques are also employed, including immersed-boundary methods \citep{ebrahimi2022computational}, particle-based methods \citep{ye2016particle}, and continuum mathematical models \citep{marcinkowska2007comparison}.
\end{itemize}

\subsection*{\textbf{3. Outlook}} 
A significant gap in the field remains our incomplete understanding of the complex interplay among biological, physiological, and rheological factors in blood. This limitation restricts the advancement of improved diagnostic tools and optimized treatment strategies. Accurately modeling the intricate, multiscale behaviors of blood flow and hemorheology—especially near boundaries and under various physiological conditions—continues to be a major challenge, hindering the development of reliable, patient-specific simulations and broader applications to other biofluids \citep{beris2021recent}.
\subsection*{3--1}
(a) \\
\includegraphics[width=0.7\linewidth]{instr-figures/compliance_plot.png}

(b)
\begin{align*}
    \mathcal{L}\{J_c(t)\}(s)&=\frac1{1000}(\frac{10}{s}-\frac{5}{s+1/4}-\frac{3}{s+1/8}) \\
\mathcal{L}\{G_r(t)\} &=\frac1{s^2\widetilde{J_c}(s)} \\
&= \frac{500(32s^2+12s+1)}{s(32s^2+38s+5)} \\
&= \frac{100}{s}+\frac{12800s+2200}{32s^2+38s+5} \\
&=\frac{100}s+\frac{12800s+2200}{32[(s+\frac{19}{32})^2-\frac{201}{1024}]} \\
&= \frac{100}s+\frac{400(s+\frac{19}{32})-168.75}{(s+\frac{19}{32})^2-\frac{201}{1024}} \\
&= \frac{100}s+400 \times\frac{s+\frac{19}{32}}{(s+\frac{19}{32})^2-\frac{201}{1024}}-381 \times \frac{\sqrt{\frac{201}{1024}}}{(s+\frac{19}{32})^2-\frac{201}{1024}} \\
G_r(t)\ &= \mathcal{L}^{-1}\{\bar{G}_r(s)\} = 100 +400e^{-\frac{19}{32}t}\cosh(0.443t)-381e^{-\frac{19}{32}t}\sinh(0.443t) \\
&=100+e^{-0.594t}(400\cosh(0.443t)-381\sinh(0.443t)) \\
&= 100+e^{-0.594t}(9.5e^{0.443t}+390.5e^{-0.443t}) \\
 &= 100 + 9.5e^{-0.151t}+390.5e^{-1.037t} \\
\end{align*}
\\
So the relaxation time now is $\tau_1=1/0.151=6.62,\ \tau_2=1/1.037=0.964$. They are much smaller than the creep relaxation time.

\bigskip
\subsection*{3--2}
(a) Say the single spring has modulus $E_1$, the K-V element has $E_2$ and $\eta_2$ parameters, respectively. The single spring has strain, stress $\epsilon_1,\ \sigma_1$, while the K-V  $\epsilon_2,\ \sigma_2$. Thus, we have
\begin{gather*}
    \sigma = \sigma_1 = \sigma_2 \\
    \epsilon = \epsilon_1 + \epsilon_2 \\
    \sigma_1 = E_1\epsilon_1,\ \dot{\sigma}_1 = E_1\dot{\epsilon}_1\\
    \sigma_2=E_2\epsilon_2+\eta_2\dot{\epsilon}_2 \\
\end{gather*}
Then, we have
\begin{align*}
    E_2\sigma_1+\eta_2\dot{\sigma}_1+E_1\sigma_2 &=E_2 E_1\epsilon_1+\eta_2 E_1 \dot{\epsilon}_1 +E_1E_2\epsilon_2 +E_1\eta_2\dot{\epsilon}_2 \\
    (E_1+E_2)\sigma+\eta_2\dot{\sigma}&=E_1 E_2 \epsilon+E_1\eta_2\dot{\epsilon} \\
\end{align*}
which is our constitutive equation.

\medskip
(b)Do the Laplace transform of the constitutive equation, and assume the initial condition $\epsilon(0)=\sigma(0)=0$, we have
$(E_1+E_2)\bar{\sigma}(s)+\eta_2 s\bar{\sigma}(s)=E_1E_2\bar{\epsilon}(s)+E_1\eta_2 s \bar{\epsilon}(s)$, \\
Thus,
\begin{align*}
    \bar{\epsilon}(s) &=\frac{E_1+E_2+s\eta_2}{E_1E_2+E_1\eta_2 s}\bar{\sigma}(s)\\
    \bar{\sigma}(s)&=\frac{[E_1E_2+E_1\eta_2 s]\bar{\epsilon}(s)}{E_1+E_2+s\eta_2}  \\
\end{align*}
While we have $\bar{\epsilon}(s)=s\bar{J}_c(s)\bar{\sigma}(s)$ and $\bar{\sigma}(s)=s\bar{G}_r(s)\bar{\epsilon}(s)$, we get
\begin{align*}
    \bar{J}_c(s)&=\frac{(\frac{E_1}{E_2}+1)\frac1s+\tau }{E_1+E_1\tau s } \\
    &= (\frac1{E_1}+\frac1{E_2})\frac1s-\frac{E_1 \tau}{E_2(E_1+E_1\tau s)} \\
    &= (\frac1{E_1}+\frac1{E_2})\frac1s-\frac{1}{E_2(\frac1{\tau}+s)} \\
    {J}_c(t) &=(\frac1{E_1}+\frac1{E_2})-\frac{1}{E_2}e^{-\frac t\tau} \\
    \bar{G}_r(s)&=\frac{\frac{E_1}{s}+E_1 \tau}{\frac{E1}{E_2}+1+\tau s} \\
    &= \frac{E_1E_2}{(E_1+E_2)s}+\frac{{E_1}^2\tau}{(E_1+E_2)(\frac{E_1}{E_2}+1+\tau s)} \\
    &= \frac{E_1E_2}{(E_1+E_2)s}\frac{{E_1}^2}{(E_1+E_2)[(\frac{E_1}{E_2}+1)\frac 1{\tau}+ s]} \\
    {G}_r(t) &=\frac{E_1E_2}{E_1+E_2}+\frac{{E_1}^2}{E_1+E_2}\exp({-\frac {(E_1+E_2)t} {E_2 \tau}})
\end{align*}
where $\tau = \frac{\eta_2}{E_2}$

\medskip
(c) The two different models follow the same response to the behavior. Say the spring in the 3-parameter Maxwell model has modulus $E'_1$, the Maxwell element has modulus and viscosity of $E'_2, \eta'_2$, and $\tau' = \eta'_2 / E'_2$, respectively. Then they have the relations:
\begin{align*}
    E'_1&=\frac{E_1 E_2}{E_1+E_2} \\
    E'_2 &= \frac{{E_1}^2}{E_1+E_2} \\
    \tau' &=\frac{E_2}{E_1+E_2}\tau \\
    \eta'_2 &=\frac{E_1^2}{(E_1+E_2)^2}\eta_2 \\
\end{align*}

\bigskip
\subsection*{3--3}
(a) \\
% Main nodes
\begin{circuitikz}
  \draw (2,4.5) to[short] (2,4);
  \draw (2,4) to[short] (0,4)
              to[spring, l=$E_0$] (0,0)  % 添加 E0 标签
              to[short] (2,0);
  \draw (2,4) to[spring, l=$E_1$] (2,2)  % 添加 E1 标签
              to[damper, l=$\eta_1$] (2,0);  % 添加 η1 标签
  \draw (2,4) to[short] (4,4)
              to[spring, l=$E_2$] (4,2)  % 添加 E2 标签
              to[damper, l=$\eta_2$] (4,0)  % 添加 η2 标签
              to[short] (2,0);
  \draw (2,0) to[short] (2,-0.5);
\end{circuitikz}

where$ E_0=10C_r,\ E_1=100C_r,\ E_2=200C_r,\ \frac{\eta_1}{E_1}=1,\ \frac{\eta_2}{E_2}=0.5$

\medskip
(b)
\begin{align*}
    E'(\omega)&=E_{\infty}+\omega\int^{\infty}_{0}\tilde{E}(t')\sin\omega t'dt' \\
    &= 10C_r+100C_r\omega\int^{\infty}_{0}(2e^{-2t'}+e^{-t'})\sin\omega t'dt' \\
    &= 10C_r +100C_r\omega(\frac{2\omega}{4+\omega^2}+\frac{\omega}{1+\omega^2}) \\
    E''(\omega)&=\omega\int^{\infty}_{0}\tilde{E}(t')\cos\omega t'dt' \\
    &=100C_r\omega\int^{\infty}_{0}(2e^{-2t'}+e^{-t'})\cos\omega t'dt' \\
    &=100C_r\omega(\frac{4}{4+\omega^2}+\frac{1}{1+\omega^2}) \\
    \tan\delta &= \frac{E''(\omega)}{E'(\omega)} \\&= \frac{100C_r\omega(\frac{4}{4+\omega^2}+\frac{1}{1+\omega^2})}{10C_r +100C_r\omega(\frac{2\omega}{4+\omega^2}+\frac{\omega}{1+\omega^2})} \\
    &=\frac{10\omega(\frac{4}{4+\omega^2}+\frac{1}{1+\omega^2})}{1+10\omega(\frac{2\omega}{4+\omega^2}+\frac{\omega}{1+\omega^2})} \\
\end{align*}
\includegraphics[width=0.7\linewidth]{instr-figures/ps3-3.png}

\bigskip
\subsection*{3--4}
(a) \\
\begin{equation*}
   \sigma(t)=10\epsilon_0+2\cdot0.2^\alpha\frac{t^{-\alpha}}{\Gamma(1-\alpha)}\epsilon_0
\end{equation*}
\\
\includegraphics[width=0.7\linewidth]{instr-figures/ps3-4-1.png}
\\
(b) \\
Doing the Laplace transform of $G_r(t)$ leads to the following:
\begin{align*}
    \bar{G}_r(s) &= \frac{10}s+2\cdot0.2^\alpha s^{\alpha-1} \\
    \bar{J}_c(s)&=\frac1{s^2(\frac{10}s+2\cdot0.2^\alpha s^{\alpha-1})} \\
    &=\frac1{s(10+2\cdot0.2^\alpha s^{\alpha})} \\
    &=\frac1{10} \times \frac{\frac{10}{2\cdot 0.2^\alpha}}{s(s^\alpha+\frac{10}{2\cdot 0.2^\alpha})} \\
    J_c(t)&=\frac1{10}[1-E_{0.2}(-\frac{10 t^\alpha}{2\cdot 0.2^\alpha})] \\
\end{align*}
where $E_\alpha(\cdot)$ gives Mittag-Leffler function, and the Laplace inverse transform is based on property $
\mathcal{L}\left\{ 1 - E_{\alpha}(-\lambda t^{\alpha}) \right\}
= \frac{\lambda}{s\left(s^{\alpha} + \lambda\right)} $
\\
The stress is $\sigma(t) = \sigma_0(\mathcal{H}(t)-\mathcal{H}(t-5))$, then the strain will be
\begin{equation*}
    \epsilon = \sigma_0 [J_c(t)\mathcal{H}(t)-J_C(t-5)\mathcal{H}(t-5)]
\end{equation*}
\includegraphics[width=0.7\linewidth]{instr-figures/ps3-4-2.png}

\medskip
(c)
 a step stress of only length $t=10$, 
 \begin{equation*}
    \epsilon = \sigma_0 [J_c(t)\mathcal{H}(t)-J_C(t-10)\mathcal{H}(t-10)]
\end{equation*}
\includegraphics[width=0.7\linewidth]{instr-figures/ps3-4-3.png}

\bigskip
\bigskip
\subsection*{3--5}
(a) $E_d = \frac43 \pi R^3\rho g[h_0-h(R)]$

\medskip
(b) \\
\includegraphics[width=0.7\linewidth]{instr-figures/ps3-5-1.png} \\
Since the dropping event doesn’t have the reverse direction of he strain, the stress-strain curve can
be seen only half the cycle of the Lissajous plot. The stored energy is from the elastic aspect of the
matertial. In the Lisaajous figure, the pure elastic material is a straight line across the origin. The
peak stored energy happens when the strain reaches its maximum. It can be interpreted, therefore,
the area between this straight line and the origin, which is the triangle area form by two black
dashed lines and x axis in the plot above. The maximum strain is B, and the slope of this straght
line equals $G'= B \cos\delta$. Thus, the total energy is $\frac12 B^2 \cos\delta$.

\medskip
(c) The dissipation energy from Lissajous plot is the area in the ellipse area. A is the intercept
on the x axis of the curve. Thus, the semi-minor axis can be approximated with $A\sqrt{2}$, while the
semi-major axis can be approximated as $\sqrt2 B$. Thus, the whole area, namely the dissipated energy,
equals $\frac12 \pi AB$

\medskip
(d) $A = B \sin \delta$, so the ratio between dissipated and stored energy is
\begin{equation*}
    \frac{E_d}{E_s}=\frac{\frac12 \pi B^2 \sin\delta}{\frac12 B^2 \cos\delta}=\pi \tan\delta \\
\end{equation*}
The real stored energy is $\frac43 \pi R^3\rho gh_0$, so
\begin{align*}
    \frac{E_d}{E_s}&=\frac{\frac43 \pi R^3\rho g[h_0-h(R)]}{\frac43 \pi R^3\rho gh_0} = 1-\frac{h(R)}{h_0}=\pi \tan\delta \\
    \tan\delta &= \frac1\pi - \frac{h(R)}{\pi h_0}
\end{align*}

\medskip
(e)From different ball size,
\begin{align*}
    \omega_{min} &=\frac1{0.025\times1}= 40Hz \\
    \omega_{max} &= \frac 1{0.025\times10^-3} =4\times10^4Hz \\
\end{align*}
So from 40 to $4\times10^4$ Hz, this material is calibrated.
\bigskip
\bigskip
Part III
\includepdf[pages=-]{part3.pdf}
